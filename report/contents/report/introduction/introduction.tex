\chapter*{Introduction}
\addcontentsline{toc}{chapter}{Introduction}
% !! All the paragraph headers are temporary and should be removed !!
\todo{Remove this text and headings before release}
\emph{All these headings will be removed in the final version, they are here now to keep track of what needs to be in the introduction.}
\paragraph{Problem}
Unreal Tournament 3 is a game with a simple objective: Capture the enemy flag. However simple this objective may seem and however trivial a task it can be for a human, this task is already quite extensive for a machine.
\paragraph{Background}
Unreal Tournament 3 is the third instalment in the Unreal Tournament series, a series of so called \emph{First Person Shooter} games. It has multiple modes of games: Death Match, killing as many enemies as you can. Team Death Match is the same, only with teams instead of everybody versus everybody. The last is Capture the Flag, in which there are two teams, each with a flag. The objective is to walk to your enemies' flag, pick it up and bring it back to your flag to capture it. We will focus on the latter category.
\paragraph{Main Question}
Our objective is to write a so-called \emph{agent} which is capable to defeat computer controlled enemies written by the game's programmers. Our team will consist of four agents who will talk to each other to maximise their efficiency. Our team will take on a team of three enemies of the highest possible level.
\paragraph{Other Questions}
\todo{Think of other questions}.
\paragraph{Structure}
\todo{Make this part prettier}
\begin{enumerate}
\item[Chapter 2] In the report, we will start with giving an outline to the requirements of this project. We will discuss the minimum functionality every bot needs to have and some advanced functionality we will implement in ours. Furthermore, we will discuss extended functionality, which we will implement when we have time to do so.

\item[Chapter 3] Next chapter we will analyse the Unreal Tournament environment and talk about different scenarios the game will present and how our bots should react in those scenarios.

\item[Chapter 4] The following chapter will discuss design decisions such as the system structure and the strategy of our bots, as well as justification for both.

\item[Chapter 5] The next chapter talks about our testing approach and what our testing brought to light.

\item[Chapter 6] The final chapter will talk about the conclusions we can make from this project and recommendations we have for future developers.
\end{enumerate}