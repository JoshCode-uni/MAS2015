\addcontentsline{toc}{chapter}{Summary}
\chapter*{Summary}
The purpose of this project was to create a multi-agent system and win a game of Capture The Flag in the Unreal Tournament 3 environment against its native bots. The main question was how to accomplish this goal. The answer to this question involved three different components. Mainly, the problem needed to be viewed from a design perspective, as well as from a technical perspective. Besides this, addressing bugs and preventing erroneous behaviour was also crucial. \\

Firstly, in order to claim victory against native bots, three distinct roles were designed. One bot in the multi-agent system was the Flag Carrier. This bot tried to take the enemy flag and return it to the friendly base. Another bot had the role of Flag Defender. This particular bot would stay at our base in order to protect our flag from enemy bots. Last but not least, our team consisted of two Roamers. Their job was improving themselves by collecting the best weapons and armour, as well as assisting the Flag Carrier and Flag Defender with their tasks. An Overlord was also used, which allowed efficient communication in the team. Furthermore, the bots are made to prefer certain weapons based on their distance to target, in order to take down enemies efficiently. \\

Secondly, an efficient way to produce code was needed to make the bots run smoothly, thus refactoring was needed. This was an essential part of the process of making a functioning team of bots, since otherwise, the processes needed to make the bots function were too resource demanding, resulting in unwanted delays. \\

Last but not least, bugs and other technical problems needed to be addressed. The game-breaking issues (such as bots crashing on the spot) had to be addressed immediately, because otherwise the bots would not be able to win games against native bots. The minor bugs did not need to be fixed immediately, as they were less important to the outcome of our matches, but had to be solved eventually. \\