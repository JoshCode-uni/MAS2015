\chapter{Program of requirements}
This chapter is focused on all the requirements and functionalities. The MoSCoW\textsuperscript{1} method is used for prioritising requirements. The functionalities are divided into three subsections, the minimum, advanced and extended functionality, after which we will also discuss excluded functionality.
\section{Prioritising Requirements}
Using the MoSCoW method (where the M stands for Must, S for Should, C for Could and W for Won't), all the minimum requirements are mentioned under the M. All the other functionalities, which are used to improve the bots, are placed under the S. All requirements that will be implemented if time allows it are mentioned C. Everything that was taken into consideration, but eventually denied because it was either too hard to implement, would take to long to implement, or was technologically unfeasible was placed under the W.\\


\section{Minimum Functionality}
The functionalities that must be present are:
\begin{itemize}
	\item Getting the enemy flag \\
		\textit{The minimum functionality is being able to grab the enemy flag and return it, eventually winning the game of Capture the Flag. Seeing that this is the objective of the project, it is our top priority.}
	\item Shooting \\
		\textit{Whenever the bot has an enemy bot in his visual, he should shoot and try to kill him.}
\end{itemize}

\section{Advanced Functionality}
Functionalities that we should have are:
\begin{itemize}
	\item Group strategy \\
		\textit{When every bot has the same task, capturing the enemy flag, they would get in each others way. Besides this, it would leave their own flag defenceless, so a strategy in which every bot has a task is beneficial. We have opted to choose for a strategy consisting of defenders, attackers and roamers.}
	\item Pickups \\
		\textit{The bots should also have the ability to pickup weapons, armour and health. This will give us the edge when fighting the native bots, increasing the chances of winning.}
	\item Flag returning \\
		\textit{Bots should return their own flag whenever it is dropped by the enemy, as it is impossible to score when the flag is not on its own base.}
	\item Distance-based weapon choice \\
		\textit{The bots should switch weapons based on the distance between them and the opponent. If they are close to each other, they should switch to a weapon with a high close-range damage and vice versa.}
\end{itemize}

\section{Extended Functionality}
Functionalities that we could have are:
\begin{itemize}
	\item An overlord \\
		\textit{This is another agent, akin to a commander, which has an overview of the game and can control the bot-agents.}
	\item Pickup strategy \\
		\textit{This would mean the bots would choose their pickups sensibly (e.g. getting health whenever they are low on health).}
	\item Following strategy \\
		\textit{Whenever one of the bots possesses the flag, other bots will follow and protect him.}
	\item Better stuck strategy \\
		\textit{An agent tasked solely with controlling the respawning of the bots. This is in order to prevent any time loss from the bots being stuck.}
\end{itemize}

\section{Excluded Functionality}
Functionalities that will not be implemented are:
\begin{itemize}
	\item An extensive overlord \\
		\textit{The idea of making a controlling agent is a good one and making sure he has as much information as possible seems like a good idea, but the excessive amount of messages would severely impair the speed of the system.}
\end{itemize}


