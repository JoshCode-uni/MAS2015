\chapter{Program of requirements}
\todo{Add literature reference for MoSCoW method, http://cdn.projectsmart.co.uk/pdf/moscow-method.pdf}
This chapter is focused on all the requirements and functionalities. The MoSCoW\textsuperscript{1} method is used for prioritising requirements. The functionalities is divided into three subsections, the minimum, advanced and extended functionality. 
\section{Prioritising Requirements}
Using the MoSCoW method, where the M stands for Must, the S for Should, the C for Could and the W for Won't, all the minimum requirements are put under the M. All the other functionalities, which are use to better the bots, are placed under the S. Everything that was taken into consideration, but eventually denied, because it was too hard to implement, was placed under the C. The rest are placed under the W.\\


\section{Minimum Functionality}
The minimum functionality is basically to grab the enemy flag and return it. And eventually you should win the game of Capture the Flag. This taken into account, it should be the number one priority. Only implementing this, the bots would definitely lose. Maybe in the beginning, when there is only one level 1 epic bot, they would have a chance at beating them. To make successful bots, there should be advanced functionalities.\\ 

\section{Advanced Functionality}
To extend the bots, they are divided into groups:
\begin{description}
	\item[Flag Carrier] One bot with the task to capture the enemy flag and return it
	\item[Roamer] Two bots with the task to eliminate the enemy team
	\item[Flag Defender] One bot with the task to defend their own flag 
\end{description}
This design choice is made, because of various reasons. One of them is when every bot has the same task, to capture the enemy flag, they would get in each others way. Also this would be very inefficient because no one would defend their own flag. The bots should also have the ability to pickup weapons, armor and health. This will give a large boost, and the chances of winning will increase. Another thing which is important, is to return the our own flag if it is dropped by the enemy. Otherwise the bots can not score any points. There also should be strategy's to pickup things. You do not want to grab health when your life is full for example. At last the bot should switch weapons based on the distance between him and the opponent. If they are close to each other, he should switch to a weapon with a high close-range damage vice versa.\\

\section{Extended Functionality}
Functionalities that we could have are:
\begin{itemize}
	\item Another agent, one who has an overview of the game and tells the other bots what to do
	\item When one of the Roamers or the Flag capturer grabs a flag, he will be followed by the other, who will defend him against enemy bots.
	\item An agent, who has only one task, to control the respawn of the bots. Sometimes, the bots get stuck, and they are idle for some time. The agent should handle this, wasting no time.
\end{itemize}
The idea to make a controlling agent is a good one. This would make the implementation more accessible, but on the other hand, the program would run very slow. The controlling agent has to send a lot of messages to the other agents, and this would cost a lot of time. There would be a significant delay in executing tasks, which would decrease our chance at winning the game. So this functionality is excluded.\\


