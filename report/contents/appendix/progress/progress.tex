\chapter{Progress report}
%\addcontentsline{toc}{chapter}{Appendix B: Progress report}

In this appendix our group coordination contract can be found, just like a discussion and reflection on our group coordination and management.

\section*{Group cooperation contract}

To make sure that we are all on the same page and that we will work together well, we have designed a group coordination contract.
\subsubsection{Task allocation}
The project roles and matching tasks that we have are as following:
\begin{itemize}
\item[-] \textbf{The Chair of Meeting} will lead meetings and prepare agendas for them.
\item[-] The \textbf{Secretary} will take minutes during the meetings.
\item[-] The \textbf{Trac Wiki Manager} will make sure that the Trac wiki stas up to date.
\item[-] The \textbf{Task Coordinator} will make sure that everyone does what they are supposed to do.
\item[-] The \textbf{Planners} will make sure that the group stays on schedule.
\end{itemize}

We divided the roles between the group members and came to this division of tasks:

\begin{description}[noitemsep]
\item[Lisette] Chair of Meeting
\item[Ashay] Secretary
\item[Joshua] Trac Wiki Manager
\item[Jelle] Task Coordinator
\item[Gavin] Planner
\item[Youp] Planner
\end{description}

\subsubsection{Decision Making}
Our goal is to look at the capabilities of the team members, and with that in mind, we'll assign tasks. If someone encounters a problem, he/she should immediately inform the other team members through WhatsApp. (i.e. problems with git, coding problems). If no-one of the team can fix the problem, he/she should contact the TA. Design choices will be discussed during meetings. If the decision is not unanimous, the majority rules. When a vote is tied, both sides of the argument will be explained and discussed further. If nobody is willing to change their vote, the Chair of Meeting will break the tie.

\subsubsection{Presence and Availability}
If a team member will be absent during a meeting, he/she should tell the others at least one week before the meeting. If he/she has a good reason for the absence, the one week before rule doesn't count. He/she should finish his/her tasks even when he/she is absent. When someone is absent, without telling, or doesn't have a good reason and/or didn't finish his/her tasks, this will be reported to the TA, he/she should bring a treat. Before reporting, a team member will notify the person in question via WhatsApp.
\subsubsection{Meetings and Schedules}
The Chair of Meeting prepares the agenda for every meeting.
The agenda should be final and shared with the group at 00:00 on Tuesday at the latest, so the group can prepare for the meeting and suggest edits when needed.
We'll have meetings with the TA on every Tuesday, and if necessary, we'll plan another meeting. If a team member encounters an issue which concerns the whole group, he/she should contact the Chair of Meeting and then it can be put on the agenda for the next meeting.

\section{Evaluation of the cooperation during the project}
We had no problems with the most important part of cooperation. Everyone was willing to do their part and to make the project a success, and there were no fights or unprofessional arguments. We got along great and all arguments were concerning content, not concerning persons or their work ethics. The working atmosphere was pleasant, which of course had a huge influence on the actual work. The absence of free-riding for example, severely boosted the motivation of all team members, as laziness would result in a guilty conscience, while working did not feel like doing someone else's work. Another good example is that the big stick, the cooperation contract, was not used once, implying the work was done in mutual harmony.

Other parts of the cooperation were not huge issues either, but some improvements could be made. This was true for communication, which could be used in a more effective manner, as well as for planning. Both of these could be applied in a more structured manner, resulting in a more effective form of collaboration. In future projects for example, it could be a good idea to split certain tasks into multiple smaller tasks beforehand. This would allow the team to monitor the progress in a better way, allowing interference earlier on, while at the same time allowing the person executing the task to keep track of the situation better, allowing him or her to ask for assistance in a timely matter.

\section{Use of Trac}
Trac came in quite handy once we started the project. Creating tickets on Trac forced us to think about the tasks we were going to focus on in the next sprint. This helped us to have structure in our work and to have an overhead about all the things that should happen before the upcoming deadlines. Overall, Trac was great for helping us manage our project. Despite this, Trac was used less towards the end of the project, as it was often quicker to resolve an issue directly. Still, this caused us to lose track of certain issues, so in future projects Trac should be used more consistently.

During this project, we gradually started understanding Trac more and more (for example the milestones and how we could adjust them), so we're quite proficient in Trac now, making it even more useful in future projects.

\newpage
\section{Individual Reflection}
<<<<<<< HEAD
In the following section we will reflect on this project individually, going in depth about what we contributed to the project, what roles we fulfilled and finally what we learned during this project.
=======
>>>>>>> b39f04ff157f87801c7a8ef5720ddf2f1b0b8d94
\subsection{Youp}
\subsubsection{Summary}
My initial task was marked as a planner, but this quite quickly shifted to other tasks. In the end, this meant that I was mostly involved in things related to coding, such as the report and the testing of the system and that my contributions to the actual code were smaller. The contributions that were made were mostly related to the flag defender.\\

In my opinion the cooperation went quite well, with everyone doing their best to make the project a success. Despite this, some unfortunate events, some related to Goal, some related to the decision of refactoring, have caused us to do not as well as we hoped we would. While there is always the chance of running into this, I think that an improvement in communication could decrease the impact of those problems, so that is definitely a lesson that was learned.\\

Overall, this project has been very educational. Maybe not in terms of information, but definitely in more generally applicable knowledge such as communication, collaboration and project management. 

\subsubsection{Reflection}
My start at the project was a little slow, mainly due to having problems with Goal. The fact that I was not able to work on the code during the first week was the reason that my initial role was that of a planner. It quickly became apparent (although this was partly expected), that having two planners was too much for the project. This meant that my role slowly shifted towards other aspects of the project. My biggest contributions can be found in this report, as I've written large parts of it, incorporated feedback from written communication and also revised the report, both the concept and the final version. Besides this, I also got the task of testing our entire system. Unfortunately, my earlier attempts (creating custom UT3 maps and trying to control enemy bots) were too time-consuming, which wasted several hours, while the forms of testing included in Goal were not sufficient in my opinion. This meant that during most meetings, I was running tests, carefully analysing the environment and code in order to find bugs and pass them to my teammates. My contributions to the actual code are not overwhelming, but can mainly be found in the Flag Defender.\\

The cooperation in our group went quite well, although in my opinion some improvements could be made in communication. Especially after the refactoring, it was not always clear what still needed to be implemented and what was finished already. The refactoring also posed some other problems, as it took quite some time and it stalled other members in their work. Still, this is hindsight, and it's always easy to look back and see things that could have been improved.\\

While I did not gather a lot of information, this project might be the single best course in knowledge gained and lessons learned. One of the main things I learned during this project is that an adequate overview of the project is absolutely essential, as it enables you to estimate what still needs to be done and allocate your resources consequently. In order to keep an adequate overview of the project, good communication is indispensable. In addition, the refactoring has taught me that having a clear structure in place is also a main priority for a project. The lack of a good architecture made refactoring necessary, which in turn caused other problems.\\
\newpage
\subsection{Joshua}
\subsubsection{Summary}
\subsubsection{Reflection}
This project has been quite the roller coaster. From critical GOAL errors to Prolog crashes, one might say we really have seen it all. My role in this project was quite varying. My initial role was \emph{Trac Wiki Manager}, however I have fulfilled a spectrum of different roles.

\noindent
\textbf{Roles}\\
In the beginning, I helped everybody set up their local repository and project, which grew into becoming the de facto \emph{Git Master}, helping everybody with their Git errors, making sure that the repository was as clean as we could get it and keeping an eye on branching, making sure that nobody \emph{ever} made a commit to the master branch.

Besides that, I feel that I took on the role of leading the development of the bots, making sure that the tickets for each week were things we really needed. Also, within this role, I wrote the Design Specifications for the bots as a sort of protocol or standard for everybody to follow. This made sure that the overall structure of our system is clear and everybody can abide by that.

\noindent
\textbf{Contributions}\\
Of course, everybody's role is to contribute to the the final product. As of the time of writing this section, the 18th of June, I have made a total of 92 commits, adding 4315 lines to the repository and removing 1795 lines for a total of 2520 lines contributed.

However, lines contributed doesn't say everything, so here is a list of things I can remember I implemented:

\begin{itemize}
\item[-] \textbf{Refactoring.} I led the refactoring effort, which I completed together with Jelle. Over the course of a few weeks, our code had gotten messier than it should have been. To fix this, we decided to refactor the entire code, because we also wanted to implement dynamic role assignment, something which wasn't possible with our old structure.
\item[-] \textbf{Fixing Errors.} Arguably the largest part of my contributions to this project, after the giant refactoring we did, was fixing errors, bugs and unexpected behaviour. I cannot remember how much time I spent in the debug mode of Eclipse executing the bots line by line to find countless errors, but it was quite a lot.
\item[-] \textbf{Structure.} As stated above, I designed the overall structure of the Multi-Agent System.
\item[-] \textbf{Report.} I made the starting layout for the report and the layout of the title page. I also wrote the introduction and the most of chapter 4.
\item[-] \textbf{Overlord.} I wrote what was supposed to be the basic part of the Overlord agent, but is still the only code in the agent so far. Together with this, I made sure that the handling of messages about roles to the agents was handled properly.
\end{itemize}
\noindent
\textbf{Improvements}\\
What went well is that after refactoring and when there was a clear structure in place, everybody started to work hard on the project. Unfortunately, this was already far into the project, so what can be improved next time is that we start with writing a structure before we start writing lacklustre code without any structure. This was the second-largest blocking element in the development of our system.

Moving on to something else that can be improved, which is out of our control and what was the largest annoyance by far, and that is expressed lightly, is \textbf{GOAL} itself. GOAL is clearly not a finished programming language, to the point that it will give critical runtime errors and crash when you have implemented the code exactly as you should have. This causes us to write workarounds, one of which was implicating our ability to develop very much. One time, a GOAL update completely broke our system because of an error not on our part and not in our ability to fix. This error in GOAL is still not fixed to this day, the 18th of June 2015, 2 weeks after it was introduced. Moreover, the GOAL plugin for Eclipse is also a disaster to work with, giving a lot of errors which are entirely not true, so even the error checking in the plugin cannot be trusted for a second. GOAL, at the moment, is way too inconsistent and this inconsistency is what impairs us to write a good Multi-Agent System.

\noindent
\textbf{Learning}\\
The largest thing to take away from this project for me is that every team needs to start with outlining the structure of the system as clear as possible before a single line of code is written. This ensures that everybody knows how to construct the system and this prevents a lot of dirty code.

Another thing I learned this project is to never use GOAL again, at least not in it's current state. It is not in a state which is ready for use in projects like this.
\subsection{Ashay}
\subsubsection{Summary}
\subsubsection{Reflection}
\newpage
\subsection{Gavin}
\subsubsection{Summary}
\subsubsection{Reflection}
I participated in this project as a member. My role varied quite a lot during the course of this project. There were quite some problems that I met which I will discuss below.\\
\\
\\
\noindent
\textbf{Contributions}\\
\paragraph{Defender}
My part of work initially included the making of the defender bot. I created the first few versions of the defender and its basic functionalities such as picking first weapon and standing on the flag. Later the job was passed on to Youp, who continued and improved the defender.
\paragraph{Ammo Strategy}
After the defender, I made the ammo strategy for the Roamers, and then participated in the refactoring. The ammo module was a result of this, which included a list of weapon ammo to compare and decide if the bot needs to pick up ammo.
\paragraph{Overlord}
There were also attempts to create extra functionalities for the overlord. The extra functions included possible role switching during the game, as well as a system where the flag carrier would call out for escort if it is carrying the flag. The functions where eventually scrapped after the decision that the existing code has to be fixed first.
\paragraph{Report}
In the last few weeks I devoted my time to making the report, making the Preface, Summary, Conclusion, Bugs and Testing part and checked the report for grammer. After the meeting with the professor of the course technical writing, I rewrote most of the report in subjective tone to address the issue that our report was using objective tone in most parts.\\\\
\\
\noindent
\textbf{Problems}\\
\paragraph{Git}
A big problem for me was git, git itself required quite sometime to learn, and merge conflicts were even a bigger problem, there were countless occasions where I had to backup stuff and invoke a hard reset to solve the problems. Whenever I wasn't fast enough with pushing commits, merge conflicts would happen and git often failed at merging. Most of the time I had to backup and hard reset it.
\paragraph{GOAL plugin}
Another thing that often caused problems for me was the GOAL plugin. The plugin had many bugs which often led to false alarms, crashes. GOAL plugin doesn't recognize modules, and has warnings all over code that ran normally. Moreover, it doesn't recognize the code in modules, representing those in text form, which defeats the purpose of using a IDE on some level.

\subsection{Jelle}
\subsubsection{Summary}
\subsubsection{Reflection}
\newpage
\subsection{Lisette}
\subsubsection{Summary}
\subsubsection{Reflection}

\section{Project evaluation}
Overall, we found the project very interesting. The use of (computer) games is always a good way to make education more interesting and usually it helps with understanding and digesting the material, but it's in our opinion especially well-tailored for the domain of artificial intelligence. Despite this, there is definitely still room for improvement.\\

\subsection{The use of Goal}
Our biggest point would be the use of Goal. As a group, we feel that it is still to early to build the entire project around a tool that is essentially still in the development cycle. We have encountered various issues regarding Goal, ranging from minor inconveniences to critical. The program refuses to run on most virtual machines, due to problems with file-paths, and was unavailable due to a corrupted update for over a week. We have also encountered unspecified or buggy behaviour on multiple occasions, just as performance issues. Besides this, semantic errors in Eclipse are often plain wrong and debugging is a real nuisance. However, taken into account the stories that have reached us regarding last year's project, we are confident that most of these issues can be fixed by next year.

Another improvement that could be made is the number of available primitives, as they are currently quite limited. This meant unorthodox, creative or advanced solutions required complex workarounds or were practically impossible to implement (look away from the flag, hide in a corner, cover behind a wall, zigzag when shot at etc.). The already available primitives were practical and easy to use though.

\subsection{Compulsory practicals}
The compulsory practicals were in our opinion unnecessary, if not counterproductive. A university is a place where, apart from applicable knowledge, other values should be taught, such as responsibility. Compulsory education and forced attendance have no place in this.

Apart from this, practicals were often inconvenient and incompatible with other obligations, such as work or extracurricular activities (especially on Friday till 17:30), and the location was often noisy, resulting in a loss of concentration. Having regular meetings could be made obligatory, but having to work for four hours at a specified time did not add anything. 





