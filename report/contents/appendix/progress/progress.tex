\chapter*{Appendix B: Progress report}
\addcontentsline{toc}{chapter}{Appendix B: Progress report}

In this appendix our group coordination contract can be found, just like a discussion and reflection on our group coordination and management.

\section*{Group cooperation contract}

To make sure that we are all on the same page and that we will work together well, we have designed a group coordination contract.\\
\\
\textbf{Task allocation} \\
The project roles and matching tasks that we have are as following: \\
 - The Chair of Meeting: will lead meetings and prepare agenda?s for them. \\
  - Minutes Secretary: will take notes during meetings. \\
 - A Trac wiki Maintainer: will make sure that the Trac wiki stays up to date. \\
 - A Task Coordinator: will make sure everyone does what they are supposed to do. \\
 - Planners: will make sure that the group stays on schedule. \\
\\
We divided the roles between the group members and came to this division of tasks: \\
 Lisette - Chair of Meeting \\
 Ashay �- Minutes Secretary \\
 Joshua -� Trac wiki Maintainer \\
 Jelle �- Task Coordinator \\
 Gavin -� Planner \\
 Youp �- Planner \\
\\
\textbf{Decision Making}\\
Our goal is to look at the capabilities of the team members, and with that in mind, we?ll assign tasks. If someone encounters a problem, he/she should immediately inform the other team members through WhatsApp. (i.e. problems with git, coding problems). If no-one of the team can fix the problem, he/she should contact the TA. Design choices will be discussed during meetings. If the decision is not unanimous, the majority rules. When a vote is tied, both sides of the argument will be explained and discussed further. If nobody is willing to change their vote, the Chair of Meeting will break the tie. \\
\\
\textbf{Presence and availability}\\
If a team member will be absent during a meeting, he/she should tell the others at least one week before the meeting. If he/she has a good reason for the absence, the one week before rule doesn?t count. He/she should finish his/her tasks even when he/she is absent. When someone is absent, without telling, or doesn?t have a good reason and/or didn?t finish his/her tasks, this will be reported to the TA, he/she should bring a treat. Before reporting, a team member will notify the person in question via WhatsApp.\\
\\
\textbf{Meetings and Schedules:}\\
The Chair of Meeting prepares the agenda for every meeting.
The agenda should be final and shared with the group at 00:00 on Tuesday at the latest, so the group can prepare for the meeting and suggest edits when needed.
We?ll have meetings with the TA on every tuesday, and if necessary, we?ll plan another meeting. If a team member encounters an issue which concerns the whole group, he/she should contact the Chair of Meeting and then it can be put on the agenda for the next meeting.

\section*{Evaluation of the cooperation during the project}
\textit{Here we will evaluate our cooperation at the end of the project.}

\section*{Use of Trac}
Trac came in quite handy once we started the project. Creating tickets on Trac forced us to think about the tasks we were going to focus on in the next sprint. This helped us to have structure in our work and to have an overhead about all the things that should happen before the upcoming deadlines. Overall, Trac was great for helping us manage our project. In a next project we would probably use Trac in the same way.\\
Although in the beginning of this project we did not fully understand all parts of Trac (for example the milestones and how you could adjust them), but now we know all about Trac so it would probably be even more useful in the future.

\section*{Individual Reflection}
\textit{At the end of the project we will all evaluate our individual contribution to the project.}

\section*{Project evaluation}
\textit{At the end of the project, we will evaluate the project (the organization, software, and so on) as a group.}

