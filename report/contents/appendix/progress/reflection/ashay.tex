\newpage
\subsection{Ashay}
\subsubsection{Summary} This project was a tough challenge. My role during the meetings was secretary, I had to write down everything that was discussed during meetings. So everyone could look back at it. \\
My main focus in the beginning was working on the report. I knew how important reports are in projects at school, and later at work. I wrote chapter two and three, and revised it according to the feedback we got. Besides coding, another role of mine was assisting, I would assist people who had big tickets, to speed up finishing it. \\
So I ended up contributing to several things. One of them was the report, I wrote chapter two and three, and I also helped revising the other ones. Every two weeks we got feedback on our draft report, and tips for making a good report. And I checked all chapters for grammar errors and structure faults, along with constantly updating my own chapters I had to write. \\
And with the coding, I worked with Lisette to create a good strategy for weapon equipment. Another one was the strategy for following an enemy and killing him, together with Jelle. These were the major ones.\\
The next time, to improve our work, we should start early with a good and clear structure of implementing things. Not doing this led to major issues, and great time loss. And this lost time, was really needed. Also splitting tickets into smaller ones. Often people, and I, got big tickets, and this slowed us down, and was hard to do on our own. With smaller tickets, this would be avoided. Another thing was GOAL, working with an unfinished programming language, was quite the challenge. With every update, problems were resolved, but they also created several new ones. This did one time crash our program, and we had to come with a lot of workarounds to get it fixed, and this was very time consuming.\\
What I have learned is to begin structured with a project, how longer you postpone this, the more problems you will get eventually. Taking a big task is also bad, it is better to have smaller tasks, which you can faster solve, and move on to the next one. And never use a language which is still in development. 

\subsubsection{Reflection}
At first, I did not think I could really contribute to this project. The Logic Based AI course did not went very well, and my confidence was very low. GOAL was also not working very well, that did not help either. My initial role was secretary, this was during meetings. \\
\noindent
\textbf{Roles}\\
In the beginning I put my focus on making the report. I thought my skills were not good enough to actually make a significant contribution in the code. But I helped with brainstorming a lot, and how we were going to handle certain problems, during the game. An other important role of mine was assisting people. I would help them with tackling their problems, and writing code.\\
\noindent
\textbf{Contributions}\\
In the end, everybody has to contribute to the final product. Everyone had different tasks, to achieve a finished product. The main tasks I had were:
\begin{itemize}
	\item[-] \textbf{Report.} I started with the report. Joshua had a good template for LaTeX, which was very useful. I eventually wrote chapter two and three, and reviewed everything which was added, to eliminate grammar errors and typo's. I also remodeled the report, when we got feedback from our teacher.
	\item[-] \textbf{Assisting.} Whenever someone needed help, because their task was too hard, or too big, I helped them, but only when my own task could wait. This helped fixing major issues, which were blocking the progress of our project.
	\item[-] \textbf{Several strategies.} Another task I had was implementing several strategies. The major ones were weapon equipment, and worked together with someone else, because it was a big task. And the other one was a strategy for following and killing an enemy.
\end{itemize} 
\noindent	
\textbf{Improvements} We thought that made a good start. Everyone was eager to work on the project. But after a couple of weeks, we saw that the whole code was messy, and you could not find a thing you were looking for. At that moment we decided to re-factor our code, and make a clear structure. This did cost us a lot of time. When the refactoring was done, everything became much simpler, and everybody started to work hard on the project. But we had not much time left to better our code. So, what can be improved the next time is to always code structured, this will improve readability and the code will be flexible.\\
\\Another thing that we could have done better is splitting the tickets. The tickets we made were to big for one person. A better thing we could have done was to put two or more people to one ticket of ours. This would speed up the process.\\
\\Also, the programming language that we had to work with, GOAL, was very, very, annoying. GOAL is not finished yet, we could clearly see that. For example, if we wrote code, GOAL would indicate that there were numerous errors in our code. I tried solving it, but it would not matter. Eventually someone told me that this was a false warning, and the code should run just fine.
Another thing, at a point we had fine working code, and we met our performance targets. Then we got an announcement that GOAL had an update, and it was mandatory to install it. After installing it, our whole program crashed. We had to make several workarounds, to make it work again, which took a lot of time. And we did not have much time left for our project.\\
\noindent \\
\textbf{Learning} I learned a couple of things from this project. Firstly, how to make a decent report. All the classes really helped me to write on a higher level, which will be very useful in the near future. The next thing I learned is to start structured when beginning a project. The sooner you start with it, the better. You do not want to end up refactoring your code at the last stages of your project. The last thing I learned is to never use a language, which is not fully developed. This will only slower the process, and can give you a bad headache.