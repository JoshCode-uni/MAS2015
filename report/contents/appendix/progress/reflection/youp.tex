\subsection{Youp}
\subsubsection{Summary}
My initial task was marked as a planner, but this quite quickly shifted to other tasks. In the end, this meant that I was mostly involved in things related to coding, such as the report and the testing of the system and that my contributions to the actual code were smaller. The contributions that were made were mostly related to the flag defender.\\

In my opinion the cooperation went quite well, with everyone doing their best to make the project a success. Despite this, some unfortunate events, some related to Goal, some related to the decision of refactoring, have caused us to do not as well as we hoped we would. While there is always the chance of running into this, I think that an improvement in planning and communication could decrease the impact of those problems, so that is definitely a lesson that was learned.\\

Overall, this project has been very educational. Maybe not in terms of information, but definitely in more generally applicable knowledge such as communication, collaboration and project management. 

\subsubsection{Reflection}
My start at the project was a little slow, mainly due to having problems with Goal. The fact that I was not able to work on the code during the first week was the reason that my initial role was that of a planner. It quickly became apparent (although this was partly expected), that having two planners was too much for the project. This meant that my role slowly shifted towards other aspects of the project. My biggest contributions can be found in this report, as I've written large parts of it, incorporated feedback from written communication and also revised the report, both the concept and the final version. Besides this, I also got the task of testing our entire system. Unfortunately, my earlier attempts (creating custom UT3 maps and trying to control enemy bots) were too time-consuming, which wasted several hours, while the forms of testing included in Goal were not sufficient in my opinion. This meant that during most meetings, I was running tests, carefully analysing the environment and code in order to find bugs and pass them to my teammates. My contributions to the actual code are not overwhelming, but can mainly be found in the Flag Defender.\\

The cooperation in our group went quite well, as everyone did their best and no one was really free-riding. The working atmosphere was great, being professional and yet light, with room for serious in-depth discussions and jokes at the same time. In my opinion though, some improvements could be made in communication. Especially after the refactoring, it was not always clear what still needed to be implemented and what was finished already. The refactoring also posed some other problems, as it took quite some time and it stalled other members in their work. This once again shows that a good planning can have a significant impact on the result of a project. Still, this is hindsight, and it's always easy to look back and see things that could have been improved.\\

Another issue that was present in our group was the lack of feedback sometimes. This resulted in me having to rewrite portions of the report that were once written and assumed to be done, while some of them still had various flaws in them. This could be improved by instead of asking for feedback in WhatsApp, assigning the task to a single person, thereby avoiding the so-called diffusion of responsibility. This issue reared his ugly head at other times as well, mainly towards the end of the project as everyone was busy already. Improvements were also made after asking people in person.\\

While I did not gather a lot of information, this project might be the single best course in knowledge gained and lessons learned. One of the main things I learned during this project is that an adequate overview of the project is absolutely essential (or at least the leader should have the overview), as it enables you to estimate what still needs to be done and allocate your resources consequently. In order to keep an adequate overview of the project, good communication is indispensable. In addition, the refactoring has taught me that having a clear structure in place is also a main priority for a project. The lack of a good architecture made refactoring necessary, which in turn caused other problems. Finally, I have also learned a lot about LaTeX, as I often had to add parts from other people to the report. This is of course a small but valuable skill for the rest of my academic career.\\