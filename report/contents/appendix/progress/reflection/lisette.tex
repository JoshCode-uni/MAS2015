\newpage
\subsection{Lisette}
\subsubsection{Summary}
To make this project work everyone needed to contribute to the product. As for me, I contributed to this project in different ways. I mainly implemented code, but I also worked on the report. Fixing bugs was a part of my work on this project as well. 
Every team member had a role to make the project run smoothly. My role for the meetings was the Chair of Meeting. I believe that everyone had the chance to speak up in the meetings that were leaded by me.
A lot of things went great but also a lot of things went wrong and could be improved. Things that went great in this project were working with git and that eventually everyone did their part when it was really needed. 
Things that can be improved are the work I did in the first weeks of this project. I probably could have done more work than what I did. Not everything was on us, though. GOAL could also be improved. I do not think that the language is perfect the way it is now, there are still a lot of bugs in it.
As with all projects, I have learned new things again. I learned to understand the language GOAL better. Also I learned to improve my social skills and talk more about the problems that I face with the tasks that I should do. Coming up with a nice game strategy and a good way to structure our code is one of the things that can be improved too.
\subsubsection{Reflection}
When this project started, I thought it would be easy. Now it turns out it was actually not easy at all and cost a lot of effort. Below I will talk about my contributions to the project and the things that can be improved and that I learned from this project.
\\[2mm]\noindent
\textbf{Contributions}\\
At the start of this project, I put some basic but needed code in the bot files, mostly for percept handling and bots getting stuck at places in the maps. After that, for a few weeks, my main focus was on the flagcarrier (a bot that was supposed to get the enemy flag and bring it to our base). When, after I think were three weeks, the defending bot still did not do anything besides killing himself and spinning on a spot (a task assigned to another team member), I implemented some code for that bot.
Furthermore, I worked on a module that focussed on equipping different sorts of weapons, depending on the distance between the bot and an enemy. However, this turned out to be very inefficient (due to all the path actions) and because our cycles ran very slowly, we decided to not use this module. Quite a waste of time in which I could have implemented something useful, but we did not know that back then.
Later on in the project, after the refactoring was done, I cleaned up and divided all the code we had between the new files. During the project, I also helped with revising and improving the report, making it meet the requirements.
When the end of the project was getting close, we were all getting a little stressed and there were not really any clear tasks to do. What I mostly did was just trying out new things in the hope that they would improve our bots and fixing small bugs in the code.
\\[2mm]\noindent
\textbf{Roles}\\
During this project, I fulfilled the role of the Chair of Meeting, preparing and leading meetings every Tuesday. I tried to involve everyone in coming up with new tickets and ideas. Speaking for myself, I would say that I participated actively in the decision making and the discussions within the group. I enjoyed working with my group.
\\[2mm]\noindent
\textbf{Improvements}\\
What went well in our project is working with git. There were almost no problems (very different from the last time that I worked with git on a project (in the course OOP Project)). Also in the end we became more of a group, which was nice, because before we did not really talk outside the meetings. I guess it took us some time to realize that we really needed to work together in order to pass this course.
I believe that we should have worked harder in the beginning of this project. We kind of stretched some tickets over multiple weeks, which might have been unnecessary if we had just put more time into the project and finished the tickets within the sprint.
The language GOAL could also use some improvements. After the refactoring, we had a lot of errors and warnings, even though the code worked just fine. These false errors and warnings caused a lot of confusion and irritation, making it hard to spot real errors.
\\[2mm]\noindent
\textbf{Learning}\\
The thing that I learned most about during this project is the Multi-Agent System language GOAL. I feel like I know a lot more about GOAL then I did during the course Logic Based Artificial Intelligence.
Furthermore, I learned that I should just ask for help when I do not know how to implement something. I did not understand how to implement the weapon equipment module, so I did not work on it, which made the ticket last very long (until eventually Joshua helped me with it), which could have been avoided easily if I just asked my team members for some help immediately. 
Also, I learned that I should not underestimate projects. At the beginning I thought that it would be fine in the end, but now that we are at the end, I am not quite sure if we are even going to make it.
In a future project I would definitely start earlier with discussing a structure for our code, instead of just start coding with the files that we have. We were way too late with refactoring our code, which caused us to be behind schedule, already in week 7, while there were three more performance targets to achieve. Thinking of a good strategy would also be a good idea to start with on the first day. That way we can work up to that strategy every sprint, instead of just making up new things to implement on the go.
